\documentclass[polish,envcountsect,10pt]{article}

   	\usepackage[T1]{fontenc}
   	\usepackage{polski}
    \usepackage{babel}
    \usepackage{subfigure}
	\usepackage{graphicx}
	\usepackage{geometry}
	\usepackage{listings}
	\usepackage{float}
	\usepackage{graphicx}

\title{Raport przejściowy}
\author{inż. Paulina Brzęcka 184701 \and inż. Marek Borzyszkowski 184266 \and inż. Wojciech Baranowski 184574}
\date{\today}
\begin{document}

\maketitle
\tableofcontents
\newpage

\section{Projekt badawczy}

\subsection{Tytuł}

Wykorzystanie obliczeń kwantowych w algorithmic trading.

\subsection{Zleceniodawca i Opiekun}

Zleceniodawcą i opiekunem projektu jest dr inż. Piotr Mironowicz.

\subsection{Uczelnia i wydział}

Politechnika Gdańska - Wydział Elektroniki, Telekomunikacji i Informatyki.

\section{Rezultaty projektu}

\subsection{Założenia początkowe i krótki opis projektu}
Algorithmic trading, czyli handel algorytmiczny, to strategia inwestycyjna polegająca na wykorzystaniu zautomatyzowanych systemów handlowych do podejmowania decyzji inwestycyjnych na rynkach finansowych. Obliczenia kwantowe mają potencjał wzmocnienia tych strategii poprzez szybsze i bardziej efektywne przetwarzanie danych rynkowych oraz analizę trendów. W ramach tego tematu zostanie zbadana możliwość zaimplementowania agenta podejmującego decyzje inwestycyjne podczas gry na giełdzie, wykorzystując obliczenia kwantowe. Agent będzie testowany na emulatorze komputera kwantowego lub rzeczywistym komputerze, a jego skuteczność będzie porównywana z wybranymi algorytmami niekorzystającymi z technologii kwantowych. Efektem projektu będzie szkic artykułu naukowego opisującego przeprowadzone badania i wnioski z nich płynące.

\subsection{Zakres wykonanych prac i ich charakterystyka}

\subsubsection{Wykonanie systematycznego przeglądu literatury}
Wyszukanie artykułów związanych z tematem algorithm trading oraz kwantowego odpowiednika w różnych czasopismach. 
Wstępna ocena artykułów, następnie wykluczanie artykułów które w ocenie badaczy odbiegają od tematu, ostatecznie wykorzystanie metody kuli śnieżnej. 


\subsubsection{Przegląd i wybór technologii}
Sprawdzenie dostępnych narzędzi do tworzenia algorytmów na komputery kwantowe. Dobór narzędzia z wyszukanych.

\subsubsection{Wyznaczenie celów na kolejny semestr}
Na podstawie zebranej wiedzy oraz pierwszych doświadczeń z tematem, 
odpowiednie wyznaczenie celów mających największą szansę na doprowadzenie pierwszej, 
acz nie finalnej wersji produktu.

\subsection{Charakterystyka pracy zespołowej}
Podczas prac badawczych korzystano z następujących narzędzi do wymiany myśli i stworzonych artefaktów:
\begin{itemize}
	\item discord do spotkań w grupie jak i z Opiekunem, 
	\item github do trzymania artefaktów wytworzonych w czasie pracy nad projektem,
	\item \LaTeX ~do pisania dokumentów.
\end{itemize}
\subsection{Osiągnięte wyniki}
W ramach projektu badawczego w tym semestrze dokonano teoretycznego zapoznania się z algorytmiką kwantową i klasyczną w kontekście handlu algorytmicznego.
Dokonano systematycznego literatury w tej dziedzinie oraz zapoznano się dostępnymi treściami w tym temacie. 
Dodatkowo dokonano analizy istniejących rozwiązań, stanowiącą podstawę do dalszych prac nad projektem.
\subsection{Rozbieżności i zmiany w realizacji projektu}
Brak.
\subsection{Postanowienia}
Postanowienia są zgodne z planami na kolejny semestr prac.
\subsection{Plany na kolejny semestr prac}
\begin{itemize}
	\item Implementacja agenta podejmującego decyzje inwestycyjne podczas gry na giełdzie, wykorzystując obliczenia kwantowe.
	\item Testowanie agenta na emulatorze komputera kwantowego lub rzeczywistym komputerze i jego skuteczność.
	\item Porównanie wyników agenta z wybranymi algorytmami niekorzystającymi z technologii	kwantowych.
\end{itemize}

\end{document}