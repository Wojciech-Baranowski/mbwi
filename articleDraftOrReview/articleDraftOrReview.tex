\documentclass[polish,envcountsect,10pt]{article}

   	\usepackage[T1]{fontenc}
   	\usepackage{polski}
    \usepackage{babel}
	\usepackage{subfigure}
	\usepackage{graphicx}
	\usepackage{geometry}
	\usepackage{listings}
	\usepackage{float}
    \usepackage{enumerate}
    \usepackage{biblatex}
    \addbibresource{bibl.bib}


	%\usepackage[hidelinks]{hyperref}

\title{Projekt badań oraz studium pilotażowe}
\author{inż. Paulina Brzęcka 184701 \and inż. Marek Borzyszkowski 184266 \and inż. Wojciech Baranowski 184574}
\date{\today}
\begin{document}

\maketitle
\tableofcontents
\newpage

\section{Projekt badawczy}

\subsection{Tytuł}

Wykorzystanie obliczeń kwantowych w algorithmic trading.

\subsection{Opiekun}

Opiekunem projektu jest dr inż. Piotr Mironowicz.

\subsection{Cele i krótki opis}

Algorithmic trading, czyli handel algorytmiczny, to strategia inwestycyjna polegająca na wykorzystaniu zautomatyzowanych systemów handlowych do podejmowania decyzji inwestycyjnych na rynkach finansowych. Obliczenia kwantowe mają potencjał wzmocnienia tych strategii poprzez szybsze i bardziej efektywne przetwarzanie danych rynkowych oraz analizę trendów. W ramach tego tematu zostanie zbadana możliwość zaimplementowania agenta podejmującego decyzje inwestycyjne podczas gry na giełdzie, wykorzystując obliczenia kwantowe. Agent będzie testowany na emulatorze komputera kwantowego lub rzeczywistym komputerze, a jego skuteczność będzie porównywana z wybranymi algorytmami niekorzystającymi z technologii kwantowych. Efektem projektu będzie szkic artykułu naukowego opisującego przeprowadzone badania i wnioski z nich płynące.

\section{Recenzja artykułu badawczego}

\subsection{Tytuł}

Tytuł artykułu brzmi: ''Portfolio Optimization Based on Funds Standardization and Genetic Algorithm''.

\subsection{Autorzy}

Autorami artykułu są: Yao-Hsin Chou, Shu-Yu Kuo oraz Yi-Tzu Lo.

\subsection{Odnośnik}

IEEE Access, wolumen 5 (strony 21885 - 21900), nr artykułu 8049455, 23 września 2017.

\subsection{Spis treści}

\begin{enumerate}[I.]
    \item  Introduction
    \item  Related studies
    \item  Background
        \begin{enumerate}[A.]
          \item Portfolio
          \item Modern portfolio theory
          \item Sharpe ratio
        \end{enumerate}
    \item  Basic concept
    \item  Proposed methodology
        \begin{enumerate}[A.]
          \item Funds standardization
          \item Portfolio optimization using GA
            \begin{enumerate}[1)]
              \item Representation
              \item Initialization
              \item Fitness calculation
              \item Selection
              \item Crossover
              \item Mutation
            \end{enumerate}
          \item Sliding windows
        \end{enumerate}
    \item  Experiment
        \begin{enumerate}[A.]
          \item Investment targets
          \item Experimental environment
          \item Parameters of GA
          \item Comparison with MPT
          \item Self-analysis
            \begin{enumerate}[1)]
              \item Training period analysis
              \item Testing period analysis
            \end{enumerate}
        \end{enumerate}
    \item  Conclusion
\end{enumerate}

\subsection{Pozycjonowanie każdego elementu projektowania i realizacji badań}

\begin{enumerate}[I.]
    \item  Introduction
    \\
    \\
    Wstęp dostarcza kontekstu dla problemu wyboru akcji i metod jego rozwiązania. Dla projektu przydatne może być zrozumienie podstawowych wyzwań w wyborze akcji i stosowania wskaźników takich jak Sharpe ratio.
    \\
    \\
    Zawiera:
        \begin{itemize}
          \item Motywację do wyboru inwestycji w akcje nad innymi opcjami inwestycyjnymi.
          \item Główne wyzwanie wyboru akcji: zrównoważenie wysokiego zwrotu i niskiego ryzyka.
          \item Wprowadzenie wskaźnika Sharpe’a jako miary wydajności akcji.
          \item Złożoność i ograniczenia nowoczesnej teorii portfelowej (MPT).
        \end{itemize}
    Braki:
        \begin{itemize}
          \item Konkretne przykłady lub studia przypadków ilustrujące początkowe wyzwania w wyborze akcji.
          \item Bardziej szczegółowe wyjaśnienie, dlaczego istniejące metody nie redukują złożoności obliczeniowej.
        \end{itemize}
    Dodatkowe informacje:
        \begin{itemize}
          \item Podkreślenie, dlaczego nowa metoda (standaryzacja funduszy) jest konieczna.
          \item Krótkie wprowadzenie do wykorzystania algorytmów genetycznych (GA) w optymalizacji portfeli.
        \end{itemize}
    \item  Related studies
    \\
    \\
    Ta sekcja jest istotna dla porównania tradycyjnych metod CI z potencjalnym zastosowaniem obliczeń kwantowych. Może dostarczyć informacji o mocnych i słabych stronach technik CI, które mogą być porównane z technologią kwantową.
    \\
    \\
    Zawiera:
        \begin{itemize}
          \item Przegląd istniejących badań nad wyborem akcji z wykorzystaniem technik inteligencji obliczeniowej (CI).
          \item Wzmianka o różnych technikach CI, takich jak obliczenia ewolucyjne, teoria rozmyta i sieci neuronowe.
        \end{itemize}
    Braki:
        \begin{itemize}
          \item Szczegółowe porównania między różnymi technikami CI i ich skutecznością.
          \item Konkretne badania lub dane wspierające wybór GA nad innymi technikami CI.
        \end{itemize}
    Dodatkowe informacje:
        \begin{itemize}
          \item Potencjalne korzyści z integracji standaryzacji funduszy z technikami CI, zwłaszcza GA, w celu poprawy optymalizacji portfela.
        \end{itemize}
    \item  Background
    \\
    \\
    Informacje na temat dywersyfikacji i zarządzania portfelem są kluczowe dla zrozumienia, jak można zoptymalizować portfel inwestycyjny. Te zasady będą również obowiązywać w kontekście obliczeń kwantowych.
        \begin{enumerate}[A.]
          \item Portfolio
            \\
            \\
            Zawiera:
                \begin{itemize}
                  \item Wyjaśnienie znaczenia wyboru akcji.
                  \item Korzyści z dywersyfikacji inwestycji w portfelu w celu redukcji ryzyka.
                \end{itemize}
            Braki:
                \begin{itemize}
                  \item Dane historyczne wspierające strategię dywersyfikacji.
                  \item Porównawcza analiza różnych strategii dywersyfikacji portfela.
                \end{itemize}
          \item Modern portfolio theory
            \\
            \\
            Zrozumienie MPT jest podstawowe dla porównania tradycyjnych metod z nowymi podejściami, takimi jak obliczenia kwantowe. Projekt może wykorzystać te teoretyczne podstawy do stworzenia bardziej zaawansowanych modeli przy użyciu obliczeń kwantowych.
            \\
            \\
            Zawiera:
                \begin{itemize}
                  \item Podstawowe zasady MPT, w tym awersję do ryzyka i równowagę między ryzykiem a zwrotem.
                  \item Formuły matematyczne dotyczące oczekiwanego zwrotu i ryzyka portfela.
                \end{itemize}
            Braki:
                \begin{itemize}
                  \item Ograniczenia MPT w rzeczywistych scenariuszach.
                  \item Praktyczne przykłady ilustrujące zastosowanie MPT.
                \end{itemize}
          \item Sharpe ratio
            \\
            \\
            Wskaźnik Sharpe’a może być używany do oceny skuteczności agenta kwantowego w porównaniu z tradycyjnymi algorytmami. Jest to przydatny wskaźnik do analizy wyników inwestycyjnych.
            \\
            \\
            Zawiera:
                \begin{itemize}
                  \item Definicję i znaczenie wskaźnika Sharpe’a.
                  \item Formułę i wyjaśnienie, jak mierzy on wydajność inwestycji.
                \end{itemize}
            Braki:
                \begin{itemize}
                  \item Przykłady z rzeczywistego świata dotyczące wykorzystania wskaźnika Sharpe’a w zarządzaniu portfelem.
                  \item Ograniczenia wskaźnika Sharpe’a.
                \end{itemize}
            Dodatkowe informacje:
                \begin{itemize}
                  \item Jak proponowana metoda zamierza wykorzystać wskaźnik Sharpe’a do lepszej optymalizacji portfela.
                \end{itemize}
        \end{enumerate}
    \item  Basic concept
        \\
        \\
        Podstawowe koncepcje dotyczące standaryzacji funduszy i optymalizacji portfela mogą być użyte jako punkt wyjścia do implementacji algorytmu kwantowego. Zrozumienie tych idei pomoże w adaptacji metod do nowego kontekstu technologii kwantowej.
        \\
        \\
        Zawiera:
            \begin{itemize}
              \item Kluczowe pomysły stojące za proponowaną metodologią, w tym standaryzację funduszy.
              \item Wprowadzenie kluczowych koncepcji bez szczegółowej implementacji.
            \end{itemize}
        Braki:
            \begin{itemize}
              \item Szczegółowe kroki implementacji.
              \item Teoretyczne tło wspierające podstawowe koncepcje.
            \end{itemize}
        Dodatkowe informacje:
            \begin{itemize}
              \item Uzasadnienie wyboru standaryzacji funduszy jako nowatorskiego podejścia.
            \end{itemize}
    \item  Proposed methodology
        \begin{enumerate}[A.]
          \item Funds standardization
                \\
                \\
                Proces standaryzacji funduszy może być przełożony na algorytmy kwantowe, gdzie kwantowa wersja standaryzacji może oferować przewagę w szybkości i efektywności.
                \\
                \\
                Zawiera:
                    \begin{itemize}
                      \item Formuły matematyczne i kroki dotyczące standaryzacji funduszy.
                      \item Wyjaśnienie, jak fundusze są alokowane i standaryzowane.
                    \end{itemize}
                Braki:
                    \begin{itemize}
                      \item Praktyczne przykłady lub studia przypadków demonstrujące standaryzację funduszy.
                      \item Ograniczenia lub wyzwania w implementacji standaryzacji funduszy.
                    \end{itemize}
          \item Portfolio optimization using GA
             \\
                \\
                Każdy z tych kroków może być adaptowany do kwantowych algorytmów genetycznych, które mogą wykorzystywać superpozycję i splątanie do przeszukiwania przestrzeni rozwiązań bardziej efektywnie.
                \\
                \\
                Zawiera:
            \begin{enumerate}[1)]
              \item \textbf{Representation:} Jak portfel jest reprezentowany w GA.
              \item \textbf{Initialization:} Metody inicjalizacji populacji GA.
              \item \textbf{Fitness calculation:} Jak obliczana jest przystosowanie każdego portfela.
              \item \textbf{Selection:} Proces selekcji do rozmnażania następnego pokolenia.
              \item \textbf{Crossover:} Techniki krzyżowania stosowane w GA.
              \item \textbf{Mutation:} Strategie mutacji wprowadzające zmienność.
            \end{enumerate}
            Braki:
            \begin{itemize}
              \item Szczegółowe algorytmy lub pseudokod dla każdego kroku.
              \item Metryki wydajności lub benchmarki dla każdego komponentu GA.
            \end{itemize}
          \item Sliding windows
            \\
            \\
            Technika przesuwanych okien może być zastosowana również w algorytmach kwantowych, zapewniając bardziej dynamiczne i adaptacyjne podejście do optymalizacji portfela.
            \\
            \\
            Zawiera:
                \begin{itemize}
                  \item Wyjaśnienie techniki przesuwanych okien, aby uniknąć nadmiernego dopasowania.
                  \item Zastosowanie przesuwanych okien w okresach testowych.
                \end{itemize}
            Braki:
                \begin{itemize}
                  \item Konkretne dane lub przykłady pokazujące skuteczność przesuwanych okien.
                \end{itemize}
            Dodatkowe informacje:
                \begin{itemize}
                  \item Potencjalne usprawnienia lub przyszłe prace nad udoskonaleniem techniki przesuwanych okien.
                \end{itemize}
        \end{enumerate}
    \item  Experiment
        \begin{enumerate}[A.]
          \item Investment targets
            \\
            \\
            Definicja celów inwestycyjnych jest istotna dla ustalenia benchmarków do porównania wydajności agenta kwantowego z tradycyjnymi metodami.
            \\
            \\
            Zawiera:
                \begin{itemize}
                  \item Opis celów inwestycyjnych użytych w eksperymentach.
                \end{itemize}
            Braki:
                \begin{itemize}
                  \item Szczegółowe kryteria wyboru tych celów.
                  \item Dane historyczne dotyczące wybranych celów.
                \end{itemize}
          \item Experimental environment
            \\
            \\
            Zrozumienie środowiska eksperymentalnego pomoże w przygotowaniu odpowiedniego emulatora komputera kwantowego lub wykorzystaniu rzeczywistego komputera kwantowego do testowania agenta.
            \\
            \\
            Zawiera:
                \begin{itemize}
                  \item Opis sprzętu i oprogramowania użytego do eksperymentów.
                \end{itemize}
            Braki:
                \begin{itemize}
                  \item Konkretne konfiguracje lub ustawienia mogące wpływać na wyniki eksperymentalne.
                \end{itemize}
          \item Parameters of GA
            \\
            \\
            Parametry te mogą być porównane z parametrami używanymi w algorytmach kwantowych, aby zobaczyć, jakie ustawienia są najbardziej efektywne w kontekście obliczeń kwantowych.
            \\
            \\
            Zawiera:
                \begin{itemize}
                  \item Parametry użyte w GA, takie jak rozmiar populacji, stopa mutacji itp.
                \end{itemize}
            Braki:
                \begin{itemize}
                  \item Uzasadnienie wyboru konkretnych wartości parametrów.
                  \item Analiza wrażliwości różnych ustawień parametrów.
                \end{itemize}
          \item Comparison with MPT
            \\
            \\
            Wyniki tej sekcji mogą dostarczyć punktu odniesienia do porównania wydajności agenta kwantowego z tradycyjnymi metodami optymalizacji portfela, takimi jak MPT.
            \\
            \\
            Zawiera:
                \begin{itemize}
                  \item Analiza porównawcza proponowanej metody z MPT.
                \end{itemize}
            Braki:
                \begin{itemize}
                  \item Szczegółowe wyniki statystyczne wspierające porównanie.
                  \item Ograniczenia obu metod w różnych warunkach rynkowych.
                \end{itemize}
          \item Self-analysis
          \\
          \\
          Analiza okresów treningowych i testowych jest kluczowa do oceny skuteczności agenta kwantowego w różnych warunkach rynkowych.
          \\
          \\
          Zawiera:
            \begin{enumerate}[1)]
              \item \textbf{Training period analysis:} Analiza wydajności podczas okresu treningowego.
              \item \textbf{Testing period analysis:} Analiza wydajności podczas okresu testowego.
            \end{enumerate}
          Braki:
            \begin{itemize}
              \item Dane dotyczące długoterminowej wydajności.
              \item Porównanie z innymi metodami optymalizacji portfela poza MPT.
            \end{itemize}
        Dodatkowe informacje:
            \begin{itemize}
              \item Wgląd w to, jak okresy treningowe i testowe wpływają na wydajność portfela.
            \end{itemize}
        \end{enumerate}
    \item  Conclusion
            \\
            \\
            Wnioski mogą dostarczyć cennych informacji na temat potencjalnych korzyści i ograniczeń zastosowania algorytmów genetycznych, co jest istotne przy porównywaniu z technologią kwantową. Dodatkowo, wnioski mogą wskazać kierunki przyszłych badań, które mogą być adaptowane do kontekstu obliczeń kwantowych.
            \\
            \\
            Zawiera:
                \begin{itemize}
                  \item Podsumowanie wyników i skuteczności proponowanej metody.
                  \item Zalety standaryzacji funduszy i GA w optymalizacji portfela.
                \end{itemize}
            Braki:
                \begin{itemize}
                  \item Szczegółowa dyskusja na temat przyszłych kierunków badań.
                  \item Potencjalne ograniczenia lub wady proponowanej metody.
                \end{itemize}
            Dodatkowe informacje:
                \begin{itemize}
                  \item Rekomendacje dla praktyków dotyczące stosowania proponowanej metody w rzeczywistych scenariuszach.
                \end{itemize}
\end{enumerate}

\subsection{Mocne strony}

\subsubsection{Nowa metodologia}

\begin{itemize}
	\item \textbf{Standaryzacja funduszy:}
    wprowadzenie standaryzacji funduszy jako nowej metody reprezentowania zwrotu portfela i obliczania ryzyka portfela jest znaczącą zaletą. Metoda ta ma na celu rozwiązanie niedoskonałości tradycyjnych obliczeń ryzyka.
	\item \textbf{Kompleksowa ocena ryzyka:}
    artykuł twierdzi, że standaryzacja funduszy może dokładnie ocenić ryzyko portfela, uwzględniając interakcje między wszystkimi akcjami, co jest bardziej kompleksowe niż tradycyjne relacje między parami akcji.
\end{itemize}

\subsubsection{Efektywność i redukcja złożoności}

\begin{itemize}
	\item \textbf{Zmniejszona złożoność obliczeń:}
    kluczową zaletą jest zmniejszenie złożoności obliczeń. Artykuł podkreśla, że złożoność proponowanej metody nie rośnie wraz z liczbą akcji, co jest znaczną poprawą w stosunku do tradycyjnych metod.
	\item \textbf{Integracja algorytmu genetycznego:}
    połączenie algorytmu genetycznego z współczynnikiem Sharpe’a i standaryzacją funduszy w celu znalezienia optymalnego portfela jest innowacyjnym podejściem, które wykorzystuje zalety algorytmów ewolucyjnych w problemach optymalizacyjnych.
\end{itemize}

\subsubsection{Praktyczne rozwiązania}

\begin{itemize}
	\item \textbf{Technika sliding window:}
    zastosowanie techniki sliding window w celu uniknięcia problemu nadmiernego dopasowania jest praktycznym rozwiązaniem powszechnego problemu w optymalizacji portfela, co zwiększa solidność metody.
	\item \textbf{Efektywne rozproszenie ryzyka:}
    wyniki eksperymentalne pokazują, że proponowana metoda może skutecznie rozproszyć ryzyko i zapewnić niższe ryzyko oraz stabilne zwroty w porównaniu do tradycyjnych metod.
\end{itemize}

\subsubsection{Walidacja eksperymentalna}

\begin{itemize}
	\item \textbf{Rozległe testowanie:}
    artykuł wspomina o testowaniu różnych okresów szkoleniowych i testowych, co sugeruje dokładną walidację proponowanej metody. Takie kompleksowe testowanie może zwiększyć wiarygodność wyników.
\end{itemize}

\subsection{Słabe strony}

\subsubsection{Złożoność artykułu}

\begin{itemize}
	\item \textbf{Techniczny żargon:}
    artykuł zawiera kilka terminów technicznych i pojęć (np. standaryzacja funduszy, algorytm genetyczny, sliding window), które mogą być trudne do zrozumienia dla czytelników bez tła w optymalizacji finansowej lub metodach obliczeniowych.
	\item \textbf{Ograniczone wyjaśnienia:}
    chociaż artykuł wprowadza standaryzację funduszy, nie dostarcza szczegółowego wyjaśnienia, jak to działa, co może utrudniać czytelnikom zrozumienie jej implementacji i zalet.
\end{itemize}

\subsubsection{Zakres i zastosowanie}

\begin{itemize}
	\item \textbf{Generalizacja wyników:}
    artykuł nie określa warunków rynkowych ani klas aktywów testowanych, co może ograniczać postrzeganą uniwersalność wyników. Czytelnicy mogą się zastanawiać, czy metoda jest równie skuteczna w różnych rynkach i warunkach ekonomicznych.
	\item \textbf{Przesadne poleganie na wynikach eksperymentalnych:}
    chociaż wyniki eksperymentalne są obiecujące, artykuł nie wspomina o teoretycznej walidacji ani porównaniu z szerszym zakresem tradycyjnych metod poza tym, co jest konieczne.
\end{itemize}

\subsubsection{Ocena i Benchmarking}

\begin{itemize}
	\item \textbf{Brak szczegółowej analizy porównawczej:}
	chociaż metoda jest porównywana z tradycyjnymi metodami, artykuł nie określa, które tradycyjne metody były używane do porównania, ani nie dostarcza szczegółów dotyczących użytych metryk wydajności lub benchmarków.
	\item \textbf{Rozważania ryzyka:}
	artykuł wspomina o niższym ryzyku i stabilnym zwrocie, ale nie dostarcza szczegółowych informacji na temat tego, jak te metryki są ilościowo oceniane lub porównywane z istniejącymi benchmarkami.
\end{itemize}

\subsubsection{Kierunki przyszłych badań}

\begin{itemize}
	\item \textbf{Brak wzmianki o przyszłych badaniach:}
	brak wzmianki o kierunkach przyszłych badań lub potencjalnych ulepszeniach może wskazywać na brak przyszłościowego myślenia lub świadomości ograniczeń proponowanej metody.
\end{itemize}

\subsection{Ocena}

Oceniając artykuł, można zauważyć, że proponowana metoda standaryzacji funduszy wraz z algorytmem genetycznym wydaje się być obiecującym podejściem do optymalizacji portfela inwestycyjnego. Artykuł przedstawia nowatorskie podejście do oceny ryzyka portfela, które może przyczynić się do zmniejszenia złożoności obliczeń i poprawy efektywności inwestycji. Jednakże istnieją pewne niedociągnięcia, takie jak brak szczegółowej analizy porównawczej z tradycyjnymi metodami oraz ograniczona ekspozycja na kierunki przyszłych badań. Pomimo tego artykuł stanowi cenny wkład w dziedzinę optymalizacji portfela, zwracając uwagę na potrzebę dalszych badań i eksperymentów w celu weryfikacji skuteczności proponowanej metody w różnych warunkach rynkowych i klasach aktywów.

\section{Wnioski}

\subsection{Wnioski końcowe z kursu}

\subsubsection{Plan i wstępne wyniki systematycznego przeglądu literatury}

Dzięki kursowi zrozumieliśmy metodologię systematycznego przeglądu literatury, nauczyliśmy się krytycznie analizować dostępne badania oraz tworzyć szczegółowy plan badawczy. Potrafimy interpretować wstępne wyniki przeglądu, co ułatwia dalsze kierowanie badaniami.

\subsubsection{Projekt badawczy i badanie pilotażowe}

Kurs nauczył nas projektowania skutecznych badań, podkreślił znaczenie badań pilotażowych oraz zarządzania procesem badawczym. Zdobyliśmy umiejętności analizy danych z badań pilotażowych i ich wykorzystania do udoskonalenia głównego badania.

\subsubsection{Recenzja artykułu badawczego}

Kurs pomógł nam w organizacji artykułu naukowego, pisaniu klarownych tekstów oraz krytycznej ocenie prac naukowych. Poznaliśmy standardy publikacyjne i procedury recenzowania, co zwiększa nasze szanse na akceptację artykułu do druku.

\section{Literatura}

\begin{itemize}
	\item Materiały wykładowe
	\item Yao-Hsin Chou, Shu-Yu Kuo i Yi-Tzu Lo. ''Portfolio Optimization Based on Funds Standarization and Genetic Algorithm". DOI: 10.1109/ACCESS.2017.2756842.
\end{itemize}

\end{document}