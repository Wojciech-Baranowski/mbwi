\documentclass[polish,envcountsect,10pt]{article}

   	\usepackage[T1]{fontenc}
   	\usepackage{polski}
    \usepackage{babel}
	\usepackage{subfigure}
	\usepackage{graphicx}
	\usepackage{geometry}
	\usepackage{listings}

	%\usepackage[hidelinks]{hyperref}
	
\title{Plan i wstępne wyniki Systematycznego Przeglądu Literatury}
\author{inż. Paulina Brzęcka 184701 \and inż. Marek Borzyszkowski 184266 \and inż. Wojciech Baranowski 184574}
\date{\today}
\begin{document}

\maketitle
\tableofcontents
\newpage

\section{Projekt badawczy}

\subsection{Tytuł}

Wykorzystanie obliczeń kwantowych w algorithmic trading.

\subsection{Opiekun}

Opiekunem projektu jest dr inż. Piotruś Mironowicz.

\subsection{Cele i krótki opis}

Algorithmic trading, czyli handel algorytmiczny, to strategia inwestycyjna polegająca na wykorzystaniu zautomatyzowanych systemów handlowych do podejmowania decyzji inwestycyjnych na rynkach finansowych. Obliczenia kwantowe mają potencjał wzmocnienia tych strategii poprzez szybsze i bardziej efektywne przetwarzanie danych rynkowych oraz analizę trendów. W ramach tego tematu zostanie zbadana możliwość zaimplementowania agenta podejmującego decyzje inwestycyjne podczas gry na giełdzie, wykorzystując obliczenia kwantowe. Agent będzie testowany na emulatorze komputera kwantowego lub rzeczywistym komputerze, a jego skuteczność będzie porównywana z wybranymi algorytmami niekorzystającymi z technologii kwantowych. Efektem projektu będzie szkic artykułu naukowego opisującego przeprowadzone badania i wnioski z nich płynące.

\section{Plan Systematycznego Przeglądu Literatury}

\subsection{Cele i pytania}

Celem jest znalezienie klasycznych algorytmów używanych w algorythmic trading, wraz z ich definicją oraz znalezienie algorytmów kwantowych mających potencjał być wykorzytanych w analogicznym celu. Następnie odpowiedzieć na pytanie który z nich najlepiej spełnia swoją rolę, poświęcając szczególną uwagę algorytmom kwantowym. 

\subsection{Słowa kluczowe}

\begin{itemize}
	\item quantum computing,
	\item algorythmic trading,
	\item quantum algorithm,
	\item quantum simulation,
	\item finance,
	\item portfolio, 
	\item optimization,
	\item market predicting,
	\item trading strategy.
\end{itemize}

\subsection{Wyszuwiwane frazy}

\begin{itemize}
	\item algorythmic trading,
	\item quantum algorythmic trading,
	\item trading strategy,
	\item quantum tradng strategy,
	\item portfolio optimization,
	\item market predicting.
\end{itemize}


\subsection{Bazy danych literatury}

\begin{itemize}
	\item IEEEXplore,
	\item arxiv,
	\item scihub.
\end{itemize}


\subsection{Kryteria inkluzji}

\begin{itemize}
	\item rok publikacji: $\ge$ 2021,
	\item język publikacji: polski, angielski,
	\item typ publikacji: artykuł, książka, praca naukowa.
\end{itemize}

\subsection{Kryteria ekskluzji}

\begin{itemize}
	\item rok publikacji: $\le$ 2020,
	\item język publikacji: inny niż polski i angielski,
	\item typ publikacji: gazeta, magazyn, czasopismo, blog.
\end{itemize}

\subsection{Kryteria jakości}

\begin{itemize}
	\item metoda badawcza: analiza danych wtórnych, eksperymenty,
	\item liczba cytowań: $\ge$ 4,
	\item dane badawcze: historyczne odczyty giełdowe,
\end{itemize}

\subsection{Ekstrakcja danych}

\begin{itemize}
	\item wykorzystane algorytmy,
	\item definicja algorytmów,
	\item rezultaty na zadanych zbiorach danych,
\end{itemize}

\subsection{Proces SLR}

\subsubsection{Kroki procesu}

\begin{enumerate}
	\item ustawienie odpowiednich filtrów w wyszukiwarkach,
	\item wykonanie wielu zapytań z wykorzystaniem różnych kombinacji słów kluczowych,
	\item filtrowanie zebranych artykułów względem kryteriów,
	\item wyekstraktowanie i uporzadkowanie danych.
\end{enumerate}

\subsubsection{Podział ról w zespole}

\begin{itemize}
	\item osoba wyszukująca dane: Paulina,
	\item osoba weryfikująca dane: Marek,
	\item osoba tworząca raport: Wojciech.
\end{itemize}

\subsubsection{Wykorzystane narzędzia}

\begin{itemize}
	\item wyżej wymienione bazy literatury,
	\item arkusz kalkulacyjny,
	\item latex,
	\item nvim,
	\item discord (medium komunikacyjne).
\end{itemize}

\section{Wyniki Systematycznego Przeglądu Literatury}

\subsection{Wyniki liczbowe}

\subsection{Atrykuły wybrane do ekstrakcji danych}

\subsection{Artykuły wybrane przy użyciu techniki snowballingu}

\subsection{Statystyki artykułów}

\subsection{Wstępnie wybrane dane}

\section{Wnioski}

\subsection{Proces SLR}

\subsection{Wyniki SLR}

\section{Literatura}

\end{document}
