\documentclass[polish,envcountsect,10pt]{article}

   	\usepackage[T1]{fontenc}
   	\usepackage{polski}
    \usepackage{babel}
	\usepackage{subfigure}
	\usepackage{graphicx}
	\usepackage{geometry}
	\usepackage{listings}
	\usepackage{float}
 \usepackage{graphicx}
 \usepackage{biblatex}
\addbibresource{scopus.bib}


	%\usepackage[hidelinks]{hyperref}

\title{Plan i wstępne wyniki Systematycznego Przeglądu Literatury}
\author{inż. Paulina Brzęcka 184701 \and inż. Marek Borzyszkowski 184266 \and inż. Wojciech Baranowski 184574}
\date{\today}
\begin{document}

\maketitle
\tableofcontents
\newpage

\section{Projekt badawczy}

\subsection{Tytuł}

Wykorzystanie obliczeń kwantowych w algorithmic trading.

\subsection{Opiekun}

Opiekunem projektu jest dr inż. Piotr Mironowicz.

\subsection{Cele i krótki opis}

Algorithmic trading, czyli handel algorytmiczny, to strategia inwestycyjna polegająca na wykorzystaniu zautomatyzowanych systemów handlowych do podejmowania decyzji inwestycyjnych na rynkach finansowych. Obliczenia kwantowe mają potencjał wzmocnienia tych strategii poprzez szybsze i bardziej efektywne przetwarzanie danych rynkowych oraz analizę trendów. W ramach tego tematu zostanie zbadana możliwość zaimplementowania agenta podejmującego decyzje inwestycyjne podczas gry na giełdzie, wykorzystując obliczenia kwantowe. Agent będzie testowany na emulatorze komputera kwantowego lub rzeczywistym komputerze, a jego skuteczność będzie porównywana z wybranymi algorytmami niekorzystającymi z technologii kwantowych. Efektem projektu będzie szkic artykułu naukowego opisującego przeprowadzone badania i wnioski z nich płynące.

\section{Plan Systematycznego Przeglądu Literatury}

\subsection{Cele i pytania}

Celem jest znalezienie klasycznych algorytmów używanych w algorythmic trading, wraz z ich definicją oraz znalezienie algorytmów kwantowych mających potencjał być wykorzystanych w analogicznym celu.
Następnie odpowiedzieć na pytanie które z nich powinny być wykorzystane w porównaniu dwóch podejść.

\subsection{Słowa kluczowe}
Słowa kluczowe są podzielone na zbiory A, B i C:
\begin{itemize}
	\item zbiór A1 zawiera (quantum),
	\item zbiór A2 zawiera (algorithm OR simulation OR \_),
	\item zbiór B zawiera (portfolio OR market predicting OR trading strategy OR finance),
	\item zbiór C zawiera (optimization OR \_),
\end{itemize}
gdzie znak \emph{\_} oznacza pusty ciąg.

\subsection{Wyszukiwane frazy}
Wyszukiwane frazy będą dzielić się na 2 grupy stworzone według następujących schematów:
\begin{enumerate}
	\item A1 AND B AND C, czyli:
	      \begin{verbatim}
		"quantum" AND ("portfolio" OR "market predicting" OR "trading strategy"
		OR "finance") AND ("optimization" OR "")},
	\end{verbatim}
	\item A2 AND B AND C, czyli:
	      \begin{verbatim}
		("algorithm" OR "simulation" OR "") AND ("portfolio" OR "market predicting"
		OR "trading strategy" OR "finance") AND ("optimization" OR "")},
	\end{verbatim}
\end{enumerate}
gdzie schemat 1 będzie wykorzystany do wyszukiwania algorytmów kwantowych, a 2 do klasycznych.

\subsection{Bazy danych literatury}

\begin{itemize}
	\item Scopus,
	\item IEEEXplore,
	\item SpringerLink.
\end{itemize}

\subsection{Kryteria inkluzji}
\begin{itemize}
	\item Rok publikacji:
	      \begin{itemize}
		      \item dla algorytmów kwantowych: $\ge$ 2021,
		      \item dla algorytmów klasycznych: $\ge$ 1990,
	      \end{itemize}
	\item język publikacji: angielski,
	\item typ publikacji: artykuł(ar), materiały z konferencji(cr),
	\item jest to pierwsze wystąpienie danego dokumentu.
\end{itemize}

\subsection{Kryteria ekskluzji}

\begin{itemize}
	\item Brak dostępu do danego dokumentu,
	\item artykuł jest mocno powiązany z innym artykułem,
	\item czasopismo artykułu jest na liście ministerialnej i ma mniej niż 70 punktów,
	\item artykuł jest opulikowany w predatory journals.
\end{itemize}

\subsection{Kryteria jakości}

\begin{itemize}
	\item Artykuł był recenzowany lub znajduje się na liście filadelfijskiej,
	\item dane badawcze są historycznymi odczytami giełdowymi.
\end{itemize}

\subsection{Ekstrakcja danych}

\begin{itemize}
	\item Wykorzystane algorytmy,
	\item definicja algorytmów,
	\item rezultaty na zadanych zbiorach danych.
\end{itemize}

\subsection{Proces SLR}

\subsubsection{Kroki procesu}

\begin{enumerate}
	\item Stworzenie planu wyszukiwania oraz dobór kryteriów,
	\item stworzenie zapytań do wyszukiwarek,
	\item wyciągnięcie danych z wyszukiwarek,
	\item dobór artykułów na podstawie kryteriów ekskluzji,
	\item dobór artykułów na podstawie tytułów i abstraktów,
	\item dobór artykułów na podstawie zawartości,
	\item sprawdzenie artykułów względem kryteriów jakości,
	\item ostateczna selekcja na podstawie jakości artykułów,
	\item dodanie do puli artykułów zawartych w referencjach artykułów już istniejących,
	\item uporządkowanie danych.
\end{enumerate}

\subsubsection{Podział ról w zespole}

\begin{table}[H]
	\caption{Podział danych etapów SLR między osobami w zespole}
	\centering
	\begin{tabular}{|p{1.7cm}|p{5cm}|p{5cm}|}
		\hline
		Krok          & Osoba wyznaczona do zadania & Recenzent                 \\
		\hline
		1, 2          & Wojciech i Marek            & Paulina                   \\
		\hline
		3             & Wojciech                    & Marek                     \\
		\hline
		4, 5, 6, 7, 8 & Paulina, Wojciech i Marek   & Paulina, Wojciech i Marek \\
		\hline
		9             & Paulina                     & Wojciech                  \\
		\hline
		10            & Paulina i Marek             & Wojciech                  \\
		\hline
	\end{tabular}
\end{table}

\subsubsection{Wykorzystane narzędzia}

\begin{itemize}
	\item Wyżej wymienione bazy literatury,
	\item arkusz kalkulacyjny,
	\item latex,
	\item nvim,
	\item discord (medium komunikacyjne).
\end{itemize}

\section{Wyniki Systematycznego Przeglądu Literatury}

Z powodu tematyki proces SLR podzielono na 2 procesy. Jeden dotyczący algorytmów klasycznych, drugi kwantowych.

\subsection{Wyniki liczbowe}

\begin{table}[H]
	\caption{Liczba wybranych publikacji w każdym z etapów SLR}
	\centering
	\begin{tabular}{|p{2cm}|p{3cm}|p{3cm}|p{3cm}|p{3cm}|}
		\hline
		Baza danych  & Liczba znalezionych pozycji & Liczba po usunięciu duplikatów & Liczba po usunięciu zgodnie z kryteriami eksluzji & Liczba po zastosowaniu kryteriów jakości i przejrzeniu artykułów \\
		\hline
		Scopus       & 468                         & 384                            & 125                                               & 30                                                               \\
		\hline
		IEEEXplore   & 83                          & 7                              & 0                                                 & 0                                                                \\
		\hline
		SpringerLink & 173                         & 44                             & 19                                                & 5                                                                \\
		\hline
	\end{tabular}
\end{table}

Warto dodać, że 11 z ostatecznie wybranych artykułów dotyczy algorytmów kwantowych, a 24 dotyczą algorytmów klasycznych.

\subsection{Artykuły wybrane do ekstrakcji danych}

\begin{table}[H]
	\caption{Lista wybranych artykułów dot. algorytmów klasycznych znalezionych w procesie SLR zgodnie z kryteriami inkluzji, ekskluzji i jakości}
	\centering
	\begin{tabular}{|p{2cm}|p{6cm}|p{3cm}|p{2cm}|}
		\hline
		Źródło & Tytuł                                                                                                                                               & Autorzy                           & Rok publikacji \\
		\hline
		scopus & The mean-variance cardinality constrained portfolio optimization problem: An experimental evaluation of five multiobjective evolutionary algorithms & Anagnostopoulos K.P.; Mamanis G.  & 2011           \\
		\hline
		scopus & Population-based algorithm portfolios for numerical optimization                                                                                    & Peng F.; Tang K.; Chen G.; Yao X. & 2010           \\
		\hline
		scopus & Using genetic algorithm to support portfolio optimization for index fund management                                                                 & Oh K.J.; Kim T.Y.; Min S.         & 2005           \\
		\hline
		scopus & A learning-guided multi-objective evolutionary algorithm for constrained portfolio optimization                                                     & Lwin K.; Qu R.; Kendall G.        & 2014           \\
		\hline
	\end{tabular}
\end{table}

\begin{table}[H]
	\caption{Lista wybranych artykułów dot. algorytmów kwantowych znalezionych w procesie SLR zgodnie z kryteriami inkluzji, ekskluzji i jakości}
	\centering
	\begin{tabular}{|p{2cm}|p{6cm}|p{3cm}|p{2cm}|}
		\hline
		Źródło       & Tytuł                                                                                                & Autorzy                                                       & Rok publikacji \\
		\hline
		scopus       & Hybrid quantum-classical optimization with cardinality constraints and applications to finance       & Fernández-Lorenzo S.; Porras D.; García-Ripoll J.J.           & 2021           \\
		\hline
		scopus       & Best practices for portfolio optimization by quantum computing, experimented on real quantum devices & Buonaiuto G.; Gargiulo F.; De Pietro G.; Esposito M.; Pota M. & 2023           \\
		\hline
		scopus       & A quantum online portfolio optimization algorithm                                                    & Lim D.; Rebentrost P.                                         & 2024           \\
		\hline
		springerlink & Recent progress and perspectives on quantum computing for finance                                    & Yehui TangJunchi YanGuoqiang HuBaohua ZhangJinzan Zhou        & 2022           \\
		\hline
	\end{tabular}
\end{table}

\subsection{Artykuły wybrane przy użyciu techniki snowballingu}

\begin{table}[H]
	\caption{Lista artykułów wybranych za pomocą techniki snowballingu}
	\centering
	\begin{tabular}{|p{2cm}|p{6cm}|p{3cm}|p{2cm}|}
		\hline
		Źródło & Tytuł                                                                              & Autorzy                                                & Rok publikacji \\
		\hline
		scopus & A fast and elitist multiobjective genetic algorithm: NSGA-II                       & Deb, K., Pratap, A., Agarwal, S., Meyarivan, T..       & 2002           \\
		\hline
		scopus & Heuristics for cardinality constrained portfolio optimisation                      & Chang, T.-J., Meade, N., Beasley, J.E., Sharaiha, Y.M. & 2000           \\
		\hline
		scopus & Portfolio optimization problems in different risk measures using genetic algorithm & Chang, T.-J., Yang, S.-C., Chang, K.-J.                & 2009           \\
		\hline
		scopus & Particle swarm optimization approach to portfolio optimization                     & Cura, T.                                               & 2009           \\
		\hline
	\end{tabular}
\end{table}

\subsection{Statystyki artykułów}

\begin{figure}[H]
	\centering
	\includegraphics[width=0.75\linewidth ]{wykres.png}
\end{figure}

\begin{figure}[H]
	\centering
	\includegraphics[width=0.75\linewidth ]{wykres2.png}
\end{figure}

\begin{figure}[H]
	\centering
	\includegraphics[width=0.75\linewidth ]{wykres3.png}
\end{figure}

\subsection{Wstępnie wybrane dane}

Na podstawie wybranych artykułów znaleziono listę algorytmów klasycznych oraz kwantowych mogących miec zastosowanie w algorithmic trading. Są to między innymi:
\begin{itemize}
	\item Algorytmy genetyczne oraz ewolucyjne,
	\item algorytm hybrydowego przeszukiwania lokalnego (hybrid local search algorithm),
	\item algorytm świetlika (Firefly algorithm),
	\item algorytmy heurystyczne,
	\item algorytm roju pszczół (artificial bee colony algorithm),
	\item algorytm wyszukiwania sąsiedztwa ze zmiennymi równoległymi,
	\item Quantum beetle antennae search
	\item kwantowy algorytm optymalizacji portfela online,
	\item NSGA-II - niedominujacy algorytm sortowania genetycznego.
\end{itemize}


\section{Wnioski}

\subsection{Proces SLR}

Proces systematycznego przeglądu literatury pozwala w sposób uporzadkowany przeanalizować i dokonać selekcji spośród wielu publikacji naukowych. Bardzo pomocne sa w tym wyszukiwarki publikacji naukowych, natomiast trzeba brać pod uwagę to, że rezultaty wyszukiwań z baz danych moga się pokrywać, co wymaga dodatkowego sprawdzenia braku duplikatów i ich usunięcia. Dodatkowo, lista wyszukanych pozycji może być ogromna, dlatego trzeba położyć nacisk na to, w jaki sposób są formułować problem, aby jak najbardziej zawęzić obszar wyszukiwań z otrzymaniem zamierzonego efektu.

\subsection{Wyniki SLR}

Na podstawie przeprowadzonego procesu SLR z 724 wyszukanych publikacji naukowych wyselekcjonowano 39 artykułów, które będą pomocne podczas realizacji projektu. Analizując wyniki otrzymane przez proces SLR można zauważyć, że informatyka kwantowa jest stosunkowo młoda dziedziną informatyki, z uwagi na to, że początki publikacji naukowych dotyczących tej dzidziny przypadają na rok 2020.

\nocite{*}
\printbibliography

\end{document}