\documentclass[polish,envcountsect,10pt]{article}

   	\usepackage[T1]{fontenc}
   	\usepackage{polski}
    \usepackage{babel}
	\usepackage{subfigure}
	\usepackage{graphicx}
	\usepackage{geometry}
	\usepackage{listings}
	\usepackage{float}


	%\usepackage[hidelinks]{hyperref}
	
\title{Plan i wstępne wyniki Systematycznego Przeglądu Literatury}
\author{inż. Paulina Brzęcka 184701 \and inż. Marek Borzyszkowski 184266 \and inż. Wojciech Baranowski 184574}
\date{\today}
\begin{document}

\maketitle
\tableofcontents
\newpage

\section{Projekt badawczy}

\subsection{Tytuł}

Wykorzystanie obliczeń kwantowych w algorithmic trading.

\subsection{Opiekun}

Opiekunem projektu jest dr inż. Piotr Mironowicz.

\subsection{Cele i krótki opis}

Algorithmic trading, czyli handel algorytmiczny, to strategia inwestycyjna polegająca na wykorzystaniu zautomatyzowanych systemów handlowych do podejmowania decyzji inwestycyjnych na rynkach finansowych. Obliczenia kwantowe mają potencjał wzmocnienia tych strategii poprzez szybsze i bardziej efektywne przetwarzanie danych rynkowych oraz analizę trendów. W ramach tego tematu zostanie zbadana możliwość zaimplementowania agenta podejmującego decyzje inwestycyjne podczas gry na giełdzie, wykorzystując obliczenia kwantowe. Agent będzie testowany na emulatorze komputera kwantowego lub rzeczywistym komputerze, a jego skuteczność będzie porównywana z wybranymi algorytmami niekorzystającymi z technologii kwantowych. Efektem projektu będzie szkic artykułu naukowego opisującego przeprowadzone badania i wnioski z nich płynące.

\section{Plan Systematycznego Przeglądu Literatury}

\subsection{Cele i pytania}

Celem jest znalezienie klasycznych algorytmów używanych w algorythmic trading, wraz z ich definicją oraz znalezienie algorytmów kwantowych mających potencjał być wykorzystanych w analogicznym celu. 
Następnie odpowiedzieć na pytanie które z nich powinny być wykorzystane w porównaniu dwóch podejść. 

\subsection{Słowa kluczowe}
Słowa kluczowe są podzielone na zbiory A, B i C:
\begin{itemize}
	\item zbiór A1 zawiera (quantum),
	\item zbiór A2 zawiera (algorithm OR simulation OR \_),
	\item zbiór B zawiera (portfolio OR market predicting OR trading strategy OR finance),
	\item zbiór C zawiera (optimization OR \_),
\end{itemize}
gdzie znak \emph{\_} oznacza pusty ciąg.

\subsection{Wyszuwiwane frazy}
Wyszukiwane frazy będą dzielić się na 2 grupy stworzone według następujących schematów:
\begin{enumerate}
	\item A1 AND B AND C, czyli: 
	\begin{verbatim} 
		"quantum" AND ("portfolio" OR "market predicting" OR "trading strategy"
		OR "finance") AND ("optimization" OR "")},
	\end{verbatim}
	\item A2 AND B AND C, czyli:
	\begin{verbatim}
		("algorithm" OR "simulation" OR "") AND ("portfolio" OR "market predicting"
		OR "trading strategy" OR "finance") AND ("optimization" OR "")},
	\end{verbatim}
\end{enumerate} 
gdzie schemat 1 będzie wykorzystany do wyszukiwania algorytmów kwantowych, a 2 do klasycznych. 

\subsection{Bazy danych literatury}

\begin{itemize}
	\item Scopus,
	\item IEEEXplore,
	\item SpringerLink,
	\item arxiv,
	\item Google scholar.
\end{itemize}

\subsection{Kryteria inkluzji}
\begin{itemize}
	\item Rok publikacji: 
	\begin{itemize}
		\item dla algorytmów kwantowych: $\ge$ 2021,
		\item dla algorytmów klasycznych: $\ge$ 1990,
	\end{itemize}
	\item język publikacji: angielski,
	\item typ publikacji: artykuł(ar), materiały z konferencji(cr),
	\item jest to pierwsze wystąpienie danego dokumentu.
\end{itemize}

\subsection{Kryteria ekskluzji}

\begin{itemize}
	\item Brak dostępu do danego dokumentu,
	\item artykuł jest mocno powiązany z innym artykułem.
	\item artykuł jest opulikowany w predatory journals.
\end{itemize}

\subsection{Kryteria jakości}

\begin{itemize}
	\item Dokument jest oparty na badaniach naukowych,
	\item artykuł był recenzowany lub znajduje się na liście filadelfijskiej,
	\item czasopismo artykułu jest na liście ministerialnej i ma co najmniej 70 punktów,
	\item dane badawcze są historycznymi odczytami giełdowymi.
\end{itemize}

\subsection{Ekstrakcja danych}

\begin{itemize}
	\item Wykorzystane algorytmy,
	\item definicja algorytmów,
	\item rezultaty na zadanych zbiorach danych.
\end{itemize}

\subsection{Proces SLR}

\subsubsection{Kroki procesu}

\begin{enumerate}
	\item Stworzenie planu wyszukiwania oraz dobór kryteriów,
	\item stworzenie zapytań do wyszukiwarek,
	\item wyciągnięcie danych z wyszukiwarek,
	\item dobór artykułów na podstawie kryteriów ekskluzji,
	\item dobór artykułów na podstawie tytułów i abstraktów,
	\item dobór artykułów na podstawie zawartości,
	\item sprawdzenie artykułów względem kryteriów jakości,
	\item ostateczna selekcja na podstawie jakości artykułów,
	\item dodanie do puli artykułów zawartych w referencjach artykułów już istniejących,
	\item uporządkowanie danych.
\end{enumerate}

\subsubsection{Podział ról w zespole}

\begin{table}[H]
    \caption{Podział danych etapów SLR między osobami w zespole}
    \centering
    \begin{tabular}{|p{1.7cm}|p{5cm}|p{5cm}|}
        \hline
        Krok & Osoba wyznaczona do zadania & Recenzent\\
        \hline
        1, 2 & Wojciech i Marek & Paulina\\
        \hline
        3 & Wojciech & Marek\\
        \hline
        4, 5, 6, 7, 8 & Paulina, Wojciech i Marek & Paulina, Wojciech i Marek\\
        \hline
        9 & Paulina & Wojciech\\
        \hline
        10 & Paulina i Marek & Wojciech\\
        \hline
    \end{tabular}
\end{table}

\subsubsection{Wykorzystane narzędzia}

\begin{itemize}
	\item Wyżej wymienione bazy literatury,
	\item arkusz kalkulacyjny,
	\item latex,
	\item nvim,
	\item discord (medium komunikacyjne).
\end{itemize}

\section{Wyniki Systematycznego Przeglądu Literatury}

\subsection{Wyniki liczbowe}

\subsection{Atrykuły wybrane do ekstrakcji danych}

\subsection{Artykuły wybrane przy użyciu techniki snowballingu}

\subsection{Statystyki artykułów}

\subsection{Wstępnie wybrane dane}

\section{Wnioski}

\subsection{Proces SLR}

\subsection{Wyniki SLR}

\section{Literatura}

\end{document}
