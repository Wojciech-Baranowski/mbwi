\documentclass[polish,envcountsect,10pt]{article}

   	\usepackage[T1]{fontenc}
   	\usepackage{polski}
    \usepackage{babel}
	\usepackage{subfigure}
	\usepackage{graphicx}
	\usepackage{geometry}
	\usepackage{listings}
	\usepackage{float}


	%\usepackage[hidelinks]{hyperref}
	
\title{Projekt badań oraz studium pilotażowe}
\author{inż. Paulina Brzęcka 184701 \and inż. Marek Borzyszkowski 184266 \and inż. Wojciech Baranowski 184574}
\date{\today}
\begin{document}

\maketitle
\tableofcontents
\newpage

\section{Projekt badawczy}

\subsection{Tytuł}

Wykorzystanie obliczeń kwantowych w algorithmic trading.

\subsection{Opiekun}

Opiekunem projektu jest dr inż. Piotr Mironowicz.

\subsection{Cele i krótki opis}

Algorithmic trading, czyli handel algorytmiczny, to strategia inwestycyjna polegająca na wykorzystaniu zautomatyzowanych systemów handlowych do podejmowania decyzji inwestycyjnych na rynkach finansowych. Obliczenia kwantowe mają potencjał wzmocnienia tych strategii poprzez szybsze i bardziej efektywne przetwarzanie danych rynkowych oraz analizę trendów. W ramach tego tematu zostanie zbadana możliwość zaimplementowania agenta podejmującego decyzje inwestycyjne podczas gry na giełdzie, wykorzystując obliczenia kwantowe. Agent będzie testowany na emulatorze komputera kwantowego lub rzeczywistym komputerze, a jego skuteczność będzie porównywana z wybranymi algorytmami niekorzystającymi z technologii kwantowych. Efektem projektu będzie szkic artykułu naukowego opisującego przeprowadzone badania i wnioski z nich płynące.


\section{Projekt Badawczy}

\subsection{Cel badania}

\subsection{Luka badawcza}

\subsection{Pytania badawcze}

\subsection{Hipotezy badawcze}

\subsection{Podmioty badawcze} 

\subsection{Operacjonalizacja}

\subsection{Metody badawcze}

\subsection{Narzędzia badawcze}

\subsection{Oczekiwane Wyniki}

\subsection{Zagrożenia wiarygodności}

\subsection{Plan badawczy}

\subsection{Cele publikacyjne}


\section{Studium Pilotażowe}

\subsection{Podmioty badawcze}

\subsection{Realizacja Studium}

\subsection{Wyniki}

\subsection{Wnioski}


\section{Wnioski}

\subsection{Projekt}

\subsection{Studium pilotażowe}


\section{Literatura}

\end{document}

